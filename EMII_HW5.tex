
\documentclass[prb,preprint]
{revtex4-1} 
% The line above defines the type of LaTeX document.
% Note that AJP uses the same style as Phys. Rev. B (prb).

% The % character begins a comment, which continues to the end of the line.

\usepackage{amsmath}  % needed for \tfrac, \bmatrix, etc.
\usepackage{amsfonts} % needed for bold Greek, Fraktur, and blackboard bold
\usepackage{graphicx} % needed for figures
\newcommand{\PRLsep}{\noindent\makebox[\linewidth]{\resizebox{0.8888\linewidth}{2pt}{$\bullet$}}\bigskip}

\begin{document}

% Be sure to use the \title, \author, \affiliation, and \abstract macros
% to format your title page.  Don't use lower-level macros to  manually
% adjust the fonts and centering.

\title{EMII Homework V Due 2014/10/21}
% In a long title you can use \\ to force a line break at a certain location.

\author{Hamilton B. Carter}
\email{hcarter333@tamu.edu} % optional
% optional second address
% If there were a second author at the same address, we would put another 
% \author{} statement here.  Don't combine multiple authors in a single
% \author statement.
\affiliation{Department of Physics, Texas A\&M University, College Station, TX 77843}
% Please provide a full mailing address here.

% See the REVTeX documentation for more examples of author and affiliation lists.

\date{\today}

%\begin{abstract}


%\end{abstract}
% AJP requires an abstract for all regular article submissions.
% Abstracts are optional for submissions to the "Notes and Discussions" section.




\maketitle % title page is now complet

%\newpage
%\section{Board 1}

%\begin{figure}[h!]
%\centering
%\includegraphics[width=5in]{board1_2014_06_24.jpg}
% Notice the width specification.  Photographs should normally have a
% resolution of approximately 300 pixels per inch when printed, that is,
% a total width of about 1000 pixels for a photo to be printed one column
% wide.  Note also that this included photo is in .jpg format even though 
% a .tiff version should be submitted for final production.
%\caption{Board 1)}
%\label{Board 1}
%\end{figure}
%\centerline{\bf EMII Homework I Due 2014/09/17}
%\bigskip
\textbf{1}
\\
\\
Show that $T^\mu_{\;\;\mu} = 0$
\\
\\
First, look at the 3 + 1 interpretation of this problem
\\
\\
$T_{ij} = \dfrac{1}{4\pi}\left(-E_iE_j - B_iB_j + \dfrac{1}{2}\delta_{ij}\left(\vec{E}^2 + \vec{B}^2\right)\right)$
\\
\\
When we restrict this spatial portion to be summed over two dummy indices, we get
\\
\\
$T_{ij} = \dfrac{1}{4\pi}\left(-\vec{E}^2 - \vec{B}^2 + \dfrac{1}{2}\vec{E}^2 + \dfrac{1}{2}\vec{B}^2\right)$
\\
$= \dfrac{1}{8\pi}\left(\vec{E}^2 + \vec{B}^2\right)$
\\
\\
The $00$ term gives
\\
\\
$T_00 = \dfrac{1}{8\pi}\left(\vec{E}^2 + \vec{B}^2\right)$
\\
\\
so
$T_{ii} + T_{00} = 0$
\\
\textbf{The much, much easier way}
\\
Write the trace as
\\
\\
$T_{\;\;\mu}^\mu= \eta_{\mu\nu}T^{\mu\nu} = \eta_{\mu\nu}\dfrac{1}{4\pi}\left(F^\mu_{\;\;\sigma} F^{\nu\sigma}-\dfrac{1}{4}\eta^{\mu\nu}F_{\sigma\lambda}F^{\sigma\lambda}\right)$
\\
\\
Then, using the property $\eta_{\mu\nu}\eta^{\mu\nu} = 4$, we arrive at 
\\
\\
$T_{\;\;\mu}^\mu= \eta_{\mu\nu}T^{\mu\nu} = \eta_{\mu\nu}\dfrac{1}{4\pi}\left(F_{\nu\sigma} F^{\nu\sigma}-F_{\sigma\lambda}F^{\sigma\lambda}\right) = 0$
\\
\\

\PRLsep
\\
\newpage
\textbf{1.b.}
\\
The definition of $*F_{\mu\nu}$ is 
\\
\\
$*F_{\mu\nu} = \dfrac{1}{2}\epsilon_{\mu\nu\rho\sigma}F^{\rho\sigma}$
\\
\\
Write down the values for $*F_{\mu\nu}$ in terms of $\vec{E}$ and $\vec{B}$.  Then, show that $*F_{\mu\nu}$ can be obtained from $F_{\mu\nu}$ by sending $\vec{E} \rightarrow -\vec{B}$ and $\vec{B} \rightarrow -\vec{E}$ 
\\
\\
We need
\\
\\
$F^{0i} = E_i$, $F^{i0} = -E_i$, and $F_{ij} = \epsilon_{ijk}B_k$
\\
\\
$*F_{00} = 0$
\\
$*F_{01} = \dfrac{1}{2}\;2B_1$, $*F_{02} = \dfrac{1}{2}\;2B_2$, $*F_{03} = \dfrac{1}{2}\;2B_3$
\\
\\
$*F_{12} = \dfrac{1}{2}\epsilon_{1203}E_3 + \dfrac{1}{2}\epsilon{1230}E_3 = F_{12} = \dfrac{1}{2}2E_3$
\\
$*F_{21} = -*F_{12}$
\\
\\
done
\\
\\
\PRLsep
\\
\newpage
\textbf{2.a.}
\\
Show that the energy momentum tensor can be written as 
\\
\\
$T_{\mu\nu} = \dfrac{1}{8\pi}\left(F_{\mu\rho}F_\nu^{\;\;\rho}+*F_{\mu\rho}*F_\nu^{\;\;\rho}\right)$
\\
\\
$*F_\nu^\rho = \eta_{\nu\alpha}*F^{\alpha\rho} = \eta_{\nu\alpha}\epsilon^{\alpha\rho\beta\gamma}F_{\beta\gamma}$
\\
\\
$*F_{\mu\rho}*F_\nu^\rho = \dfrac{1}{4}\eta_{\nu\alpha}*F^{\alpha\rho} = \eta_{\nu\alpha}\epsilon_{\mu\rho\kappa\sigma}\epsilon^{\alpha\rho\beta\gamma}F^{\kappa\sigma}F_{\beta\gamma}$
\\
\\
$= \dfrac{1}{4}\eta_{\nu\alpha}\left[-\delta_\mu^\alpha \delta_\kappa^\beta \delta_\sigma^\gamma - \delta_\mu^\beta \delta_\mu^\gamma \delta_\sigma^\alpha - \delta_\mu^\gamma \delta_\kappa^\alpha \delta_\sigma^\beta + \delta_\mu^\beta \delta_\kappa^\alpha \delta_\sigma^\gamma + \delta_\mu^\alpha \delta_\kappa^\gamma \delta_\sigma^\beta + \delta_\mu^\gamma \delta_\kappa^\beta \delta_\sigma^\alpha \right]F^{\kappa\sigma}F_{\beta\gamma}$
\\
$= \dfrac{1}{4}\left[-\eta_{\nu\mu}F^{\beta\sigma}F_{\beta\sigma} - \eta_{\nu\sigma}F^{\kappa\alpha}F_{\mu\kappa} - \eta_{\nu\kappa}F^{\kappa\beta}F_{\beta\mu} + \eta_{\nu\kappa}F^{\kappa\sigma}F_{\mu\sigma} + \eta_{\nu\mu}F^{\gamma\sigma}F_{\sigma\gamma} + \eta_{\nu\sigma}F^{\beta\sigma}F_{\beta\mu} \right]$
\\
\\
$ = \dfrac{1}{2}\eta_{\nu\mu}F_{\sigma\kappa}F^{\kappa\sigma} + \dfrac{1}{2}\eta_{\nu\sigma}F^{\kappa\sigma}F_{\kappa\mu} + \dfrac{1}{2}\eta_{\nu\kappa}F^{\kappa\sigma}F_{\mu\sigma}$
\\
$ = \dfrac{1}{2}\eta_{\nu\mu}F_{\sigma\kappa}F^{\kappa\sigma} + F_{\sigma\mu}F^{\sigma}_{\;\;\nu}$
\\
\\
$T_{\mu\nu} = \dfrac{1}{8\pi}\left[F_{\mu\nu}F_\nu^{\;\;\rho} - \dfrac{1}{2}\eta_{\mu\nu}F_{\lambda\sigma}F^{\lambda\sigma}+F_{\mu\nu}F_\nu^{\;\;\rho}\right]$
\\
$ = \dfrac{1}{4\pi}F_{\mu\nu}F_\nu^{\;\;\rho} - \dfrac{1}{16\pi}\eta_{\mu\nu}F_{\lambda\sigma}F^{\lambda\sigma}$
\\
\PRLsep
\newpage
\textbf{2.b.}
\\
Show that the energy momentum tensor for the EM field satisfies
\\
\\
$T_{\mu\rho}T^{\nu\rho} = \dfrac{1}{\left(8\pi\right)^2}\left[\left(E^2 - B^2\right) + 2\left(\vec{E}\cdot\vec{B}\right)^2\right]\delta_\mu^\nu$
\\
\\
$T_{\mu\rho}$ and $T^{\nu\rho}$ are symmetric and therefore diagonalizable.  The product of two diagonal matrices is also diagonal,therefore the result is proportional to $\delta_\mu^\nu$
\\
\\
While the above shows that we're looking for an expression that is proportional to a Kronecker delta, it doesn't tell us what the terms are.  To find that, let's look at a 3+1 decomposition of the contraction we're studying.  We have the following representations of $T^{\mu\nu}$
\\
\\
$T^{i0} = \dfrac{1}{4\pi}\left(F^0_\rho F^{i\rho}\right)$
\\
\\
$T^{ij} = \dfrac{1}{4\pi}\left(F^i_\rho F^{j\rho} - \dfrac{1}{4}\eta^{ij} F^{\rho\sigma}F_{\rho\sigma}\right)$
\\
\\
We can 3+1 deompose the entire contraction as 
\\
\\
$T^{\mu\rho} T_{\nu\rho} = T^{00} T_{00} + T^{0k} T_{0k} + T^{\mu\rho} T_{\nu\rho}$
\\
\\
\\
\PRLsep
\newpage
\textbf{3.a.}
\\
\\
Derive the equations of motion from the Lagrangian density
\\
\\
$L = -\dfrac{1}{16\pi}F^{\mu\nu}F_{\mu\nu}-\dfrac{m^2}{8\pi}A^\mu A_\mu + J^\mu A_\mu$
\\
\\
$S = \int L d^4x = -\dfrac{1}{16\pi}F^{\mu\nu}F_{\mu\nu}-\dfrac{m^2}{8\pi}A^\mu A_\mu + J^\mu A_\mu d^4x$
\\
\\
$= \int -\dfrac{1}{8\pi}F^{\mu\nu}\delta F_{\mu\nu}-\dfrac{m^2}{4\pi}A^\mu \delta A_\mu + J^\mu \delta A_\mu d^4x$
\\
\\
$= \int -\dfrac{1}{8\pi}F^{\mu\nu}\delta \left(\partial_\mu \delta A_\nu - \partial_\nu \delta A_\mu\right)-\dfrac{m^2}{4\pi}A^\mu \delta A_\mu + J^\mu \delta A_\mu d^4x$
\\
\\
$= \int \partial_\mu \left( -\dfrac{1}{4\pi}F^{\mu\nu}\delta A_\nu d\Sigma_\nu \right) + \int  \left( -\dfrac{1}{4\pi}\partial_\mu F^{\mu\nu} - \dfrac{m^2}{4\pi} A^\nu + J^\nu \delta A_\nu + J^\nu \right) \delta A_\nu d^4x$
\\
\\
The first term goes to zero because the fields vanish on the infinite spatial boundary of the surface. So, 
\\
\\
$= \int  \left( -\dfrac{1}{4\pi}\partial_\mu F^{\mu\nu} - \dfrac{m^2}{4\pi} A^\nu + J^\nu \right) \delta A_\nu d^4x$
\\
\\
and for $\delta S = 0$, we have
\\
\\
$\partial_\mu F^{\mu\nu} - \dfrac{m^2}{4\pi} A^\nu + 4\pi J^\nu = 0$
\\
\\
Taking $\partial_nu$ of the expression gives us an identity we'll need in the next step
\\
\\
$\partial_\nu \left(\partial_\mu F^{\mu\nu} - m^2 A^\nu + 4\pi J^\nu\right) = 0$
\\
\\
but, 
\\
\\
$\partial_\mu F^{\mu\nu} + 4\pi J^\nu = 0$, so
\\
\\
$- m^2 \partial_\nu A^\nu = 0$
\\
\\
$\partial_\mu \left(\partial^\mu A^\nu - \partial^\nu A^\mu\right) - \dfrac{m^2}{4\pi} A^\nu + 4\pi J^\nu = 0$
\\
\\
$ \partial_\mu\partial^\mu A^\nu - \partial_\mu\partial^\nu A^\mu - \dfrac{m^2}{4\pi} A^\nu + 4\pi J^\nu = 0$
\\
\\
Using the identity above, the second term goes to zero
\\
\\
$= \partial_\mu\partial^\mu A^\nu - \dfrac{m^2}{4\pi} A^\nu + 4\pi J^\nu = 0$
\\
\\
$= \left(\Box - \dfrac{m^2}{4\pi}\right) A^\nu + 4\pi J^\nu = 0$
\\
\\
Using the radial symmetry and the static nature of a point charge specified in the problem gives the following procession
\\
\\
$= \left(-\dfrac{\partial^2}{\partial t^2} + \dfrac{\partial^2}{\partial x_i^2} - \dfrac{m^2}{4\pi}\right) A^\nu + 4\pi J^\nu = 0$
\\
\\
$= \left(\dfrac{\partial^2}{\partial x_i^2} - \dfrac{m^2}{4\pi}\right) A^\nu + 4\pi J^\nu = 0$
\\
\\
$= \left(\dfrac{\partial^2}{\partial r^2} - \dfrac{m^2}{4\pi}\right) \phi\left(r\right) + 4\pi \rho = 0$
\\
\\
The solution for this equation can be obtained by using the Greens function for the Helholtz operator and is 
$\phi = \dfrac{q}{r}e^{-mr}$
\\
\\
Where q is the charge of the point particle and m is its mass?
\\
\\
Let's run a quick check of our 'ansatz'.  We know that
\\
\\
$\nabla^2\left(\dfrac{q}{r}e^{-mr}\right) = -4\pi q \delta^3(r)$
\\
\\
Where I've used $r$ to represent $|\vec{r}-\vec{r^\prime}|$ since the charge is at the origin.
\\
\\
Applying the product rule we get
$\nabla^2\left(\dfrac{q}{r}e^{-mr}\right) = -4\pi q \delta^3\left(r\right) e^{-mr} + \dfrac{d}{dr}\left(\dfrac{-mq}{r}e^{-mr}\right)$
\\
\\
The exponential in the first term evaluates to 1 because of the delta function.  The end result is 
$= -4\pi q \delta^3\left(r\right) + \dfrac{m^2q}{r}e^{-mr}$
\\
\\
So, 
\\
\\
$\nabla^2\left(\dfrac{q}{r}e^{-mr}\right) - m^2\dfrac{q}{r}e^-{mr} = -4\pi q \delta^3\left(r\right)$
\\
\\
\\textbf{A More Complete Solution}
\\
But, where did the q above come from?  For that matter, where did the solution come from?  Sure, it was a 'clever ansatz', but could it be constructued in and of itself.  Yup!  Here goes.
\\
\\
First, write down the first term of the differential equation more completely and far more correctly as 
\\
\\
$\nabla^2\phi = \dfrac{1}{r^2}\dfrac{\partial}{\partial r} \left(r^2\dfrac{\partial\phi}{\partial r}\right)$
\\
\\
There are other terms to $\nabla^2$ in spherical coordinates, but because of our spherical symmetry, they don't matter.  We re-write the above as 
\\
\\
$\nabla^2\phi = \dfrac{1}{r}\dfrac{\partial^2}{\partial r^2} \left(r\phi\right)$
\\
\\
Now, we can multiply the whole thing thorugh by r to get 
\\
\\
$\dfrac{d^2}{d r^2}\left(r\phi\right) - mr\phi = 0$
\\
\\
or
\\
\\
$\dfrac{d^2}{d r^2}\psi- mr\psi = 0$
\\
\\
The solution that fits the bill is 
\\
\\
$\psi \sim Ce^{\pm mr}$
\\
\\
When we remember that we multiplied through by r in the first place, we can take the r back out to get 
\\
\\
$\phi \sim \dfrac{C}{r}e^{\pm mr}$
\\
\\
And now, we need to get the $C$ to be $q$.  Retrning back to electrosttics, we have
\\
\\
$\nabla^2\phi = -4\pi\rho$
\\
\\
but, it's equal to 0 away from the actual charge
\\
\\
$\nabla^2\phi = 0$
\\
\\
This solution is simlar in that 
\\
\\
$\phi = \dfrac{C}{r}$
\\
\\
To figure out the value of $C$ we start taking integrals
\\
\\
$\int_V \nabla^2 \phi d^3x = \int_S \vec{\nabla}\phi\cdot d\vec{S} = -C\int_S \dfrac{\vec{r}}{r^3}\cdot d\vec{S} = -C\int_S \dfrac{\vec{n}}{r^2}\cdot d\vec{S}$
\\
\\
$\vec{n}\cdot d\vec{S} = sin\theta d\theta\phi r^2$
\\
\\
Which with spherical symmetry works out to $-C4\pi$
\\
\\
Now, we integrate the source charge density which is just a point charge at the origin
\\
\\
$\int_V -4\pi q \delta^2\left(\vec{r}\right) = -4\pi q$
\\
\\
so, $c = q$.
\\
\\
Since we've solved our equation of motion for the Proca equation everywhere but the origin above, we only need to look at the origin now.  At the origin, $e^{mr} = 1$ and it falls away leaving us with the same arguments as the electrostatic case and the same conculsion $C = q$.
\\
\\
\\
\\
\\
\\
\textbf{Question}
\\
Does the mass correspond to the particle or the field?  I think it's the field.  Do we need to be able to solve the differential equation independently?  If so, how?
\\
\PRLsep
\\
\\
\\
\\

% If your manuscript is conditionally accepted, the editors will ask you to
% submit your editable LaTeX source file.  Before doing so, you should move
% all tables and figure captions to the end, as shown below.  Tables come 
% first, followed by figure captions (with figure inclusions commented-out).
% Figures should be submitted as separate files, collected with the
% LaTeX file into a single .zip archive.

%\newpage   % Start a new page for tables

%\begin{table}[h!]
%\centering
%\caption{Elementary bosons}
%\begin{ruledtabular}
%\begin{tabular}{l c c c c p{5cm}}
%Name & Symbol & Mass (GeV/$c^2$) & Spin & Discovered & Interacts with \\
%\hline
%Photon & $\gamma$ & \ \ 0 & 1 & 1905 & Electrically charged particles \\
%Gluons & $g$ & \ \ 0 & 1 & 1978 & Strongly interacting particles (quarks and gluons) \\
%Weak charged bosons & $W^\pm$ & \ 82 & 1 & 1983 & Quarks, leptons, $W^\pm$, $Z^0$, $\gamma$ \\
%Weak neutral boson & $Z^0$ & \ 91 & 1 & 1983 & Quarks, leptons, $W^\pm$, $Z^0$ \\
%Higgs boson & $H$ & 126 & 0 & 2012 & Massive particles (according to theory) \\
%\end{tabular}
%\end{ruledtabular}
%\label{bosons}
%\end{table}

%\newpage   % Start a new page for figure captions

%\section*{Figure captions}

%\begin{figure}[h!]
%\centering
%\includegraphics{GasBulbData.eps}   % This line stays commented-out
%\caption{Pressure as a function of temperature for a fixed volume of air.  
%The three data sets are for three different amounts of air in the container. 
%For an ideal gas, the pressure would go to zero at $-273^\circ$C.  (Notice
%that this is a vector graphic, so it can be viewed at any scale without
%seeing pixels.)}

%\label{gasbulbdata}
%\end{figure}

%\begin{figure}[h!]
%\centering
%\includegraphics[width=5in]{ThreeSunsets.jpg}   % This line stays commented-out
%\caption{Three overlaid sequences of photos of the setting sun, taken
%near the December solstice (left), September equinox (center), and
%June solstice (right), all from the same location at 41$^\circ$ north
%latitude. The time interval between images in each sequence is approximately
%four minutes.}
%\label{sunsets}
%\end{figure}

\end{document}
