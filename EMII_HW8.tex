
\documentclass[prb,preprint]
{revtex4-1} 
% The line above defines the type of LaTeX document.
% Note that AJP uses the same style as Phys. Rev. B (prb).

% The % character begins a comment, which continues to the end of the line.

\usepackage{amsmath}  % needed for \tfrac, \bmatrix, etc.
\usepackage{amsfonts} % needed for bold Greek, Fraktur, and blackboard bold
\usepackage{graphicx} % needed for figures
\newcommand{\PRLsep}{\noindent\makebox[\linewidth]{\resizebox{0.8888\linewidth}{2pt}{$\bullet$}}\bigskip}

\begin{document}

% Be sure to use the \title, \author, \affiliation, and \abstract macros
% to format your title page.  Don't use lower-level macros to  manually
% adjust the fonts and centering.

\title{EMII Homework VIII Due 2014/11/19}
% In a long title you can use \\ to force a line break at a certain location.

\author{Hamilton B. Carter}
\email{hcarter333@tamu.edu} % optional
% optional second address
% If there were a second author at the same address, we would put another 
% \author{} statement here.  Don't combine multiple authors in a single
% \author statement.
\affiliation{Department of Physics, Texas A\&M University, College Station, TX 77843}
% Please provide a full mailing address here.

% See the REVTeX documentation for more examples of author and affiliation lists.

\date{\today}

%\begin{abstract}


%\end{abstract}
% AJP requires an abstract for all regular article submissions.
% Abstracts are optional for submissions to the "Notes and Discussions" section.




\maketitle % title page is now complet

%\newpage
%\section{Board 1}

%\begin{figure}[h!]
%\centering
%\includegraphics[width=5in]{board1_2014_06_24.jpg}
% Notice the width specification.  Photographs should normally have a
% resolution of approximately 300 pixels per inch when printed, that is,
% a total width of about 1000 pixels for a photo to be printed one column
% wide.  Note also that this included photo is in .jpg format even though 
% a .tiff version should be submitted for final production.
%\caption{Board 1)}
%\label{Board 1}
%\end{figure}
%\centerline{\bf EMII Homework I Due 2014/09/17}
%\bigskip
Link to the homework that's inspiring the notes below 
\\
\\
{\url http://people.physics.tamu.edu/pope/EM611/Prob14/prob8.pdf}
\\
\\
\textbf{1.a.}
\\
Find the TE solutions for a rectangular waveguide as shown in class.  For TE modes, we have the following
\\
\\
$E_z = 0$, $\dfrac{\partial B_z}{\partial n} = 0$ on the surface of the waveguide, and 
\\
\\
$\partial^2 B_z = \Omega^2 B_z = 0$
\\
\\
denote $B_z = \psi$
\\
\\
We know that 
\\
\\
$\dfrac{\partial \psi}{\partial n} = 0$ at $x = 0$ and $x = a$ as well as at $y = 0$ and $y = b$
\\
\\
From the class notes, we know that we want $X\left(x\right) \sim sin$ or $X\left(x\right) \sim cos$.  Since we want the deriviative of our function to be 0 at the boundary, we'll select $cos$.  The same reasoning holds for $Y\left(y\right)$
\\
\\
$X\left(x\right) \sim cos\left(\alpha x\right)$,  $Y\left(y\right) \sim cos\left(\alpha x\right)$
\\
\\
Just as in the TM mode, we get
\\
\\
$\alpha = \dfrac{m\pi}{a}$ and $\beta = \dfrac{n\pi}{b}$.
\\
\\
The final solution is 
$\psi_{mn} \sim cos\left(\dfrac{m\pi x}{a}\right)cos\left(\dfrac{n\pi x}{b}\right)$
\\
\\
also, just like the class notes
\\
\\
$\Omega^2_{mn} = \dfrac{m^2\pi^2}{a^2} + \dfrac{n^2\pi^2}{b^2}$
\\
\\
\textbf{1.b.}
\\
\\
Notice, that for the TE solutions, either $a$ or $b$ can be equal to zero and still maintain a non-trivial solution unlike in the TM case where neither $a$ or $b$ could be 0.  This gives us a slightly different expression for the minimum frequency 
\\
\\
$\omega_{minTE} = \sqrt{\dfrac{\pi^2}{a^2}}$
\\
\\
for the ratio of TE to TM cutoff frequencies, we have, assuming $a > b$,
\\
\\
$\dfrac{\omega_{minTE}}{\omega_{minTM}} = \dfrac{\pi/a}{\sqrt{\dfrac{\pi^2}{a^2} + \dfrac{\pi^2}{b^2}}}$
\\
\\
\PRLsep
\\
\newpage
\textbf{2}
\\
\\
For this problem, we'll frame the TE and TM solutions in terms of eigenfunctions and eigen values like:
\\
\\
TE:
\\
\\
$\partial_\perp^2 \psi = -\Omega^2 \psi$
\\
\\
\\
\\
\\
\\
\PRLsep
\\
\\

% If your manuscript is conditionally accepted, the editors will ask you to
% submit your editable LaTeX source file.  Before doing so, you should move
% all tables and figure captions to the end, as shown below.  Tables come 
% first, followed by figure captions (with figure inclusions commented-out).
% Figures should be submitted as separate files, collected with the
% LaTeX file into a single .zip archive.

%\newpage   % Start a new page for tables

%\begin{table}[h!]
%\centering
%\caption{Elementary bosons}
%\begin{ruledtabular}
%\begin{tabular}{l c c c c p{5cm}}
%Name & Symbol & Mass (GeV/$c^2$) & Spin & Discovered & Interacts with \\
%\hline
%Photon & $\gamma$ & \ \ 0 & 1 & 1905 & Electrically charged particles \\
%Gluons & $g$ & \ \ 0 & 1 & 1978 & Strongly interacting particles (quarks and gluons) \\
%Weak charged bosons & $W^\pm$ & \ 82 & 1 & 1983 & Quarks, leptons, $W^\pm$, $Z^0$, $\gamma$ \\
%Weak neutral boson & $Z^0$ & \ 91 & 1 & 1983 & Quarks, leptons, $W^\pm$, $Z^0$ \\
%Higgs boson & $H$ & 126 & 0 & 2012 & Massive particles (according to theory) \\
%\end{tabular}
%\end{ruledtabular}
%\label{bosons}
%\end{table}

%\newpage   % Start a new page for figure captions

%\section*{Figure captions}

%\begin{figure}[h!]
%\centering
%\includegraphics{GasBulbData.eps}   % This line stays commented-out
%\caption{Pressure as a function of temperature for a fixed volume of air.  
%The three data sets are for three different amounts of air in the container. 
%For an ideal gas, the pressure would go to zero at $-273^\circ$C.  (Notice
%that this is a vector graphic, so it can be viewed at any scale without
%seeing pixels.)}

%\label{gasbulbdata}
%\end{figure}

%\begin{figure}[h!]
%\centering
%\includegraphics[width=5in]{ThreeSunsets.jpg}   % This line stays commented-out
%\caption{Three overlaid sequences of photos of the setting sun, taken
%near the December solstice (left), September equinox (center), and
%June solstice (right), all from the same location at 41$^\circ$ north
%latitude. The time interval between images in each sequence is approximately
%four minutes.}
%\label{sunsets}
%\end{figure}

\end{document}
