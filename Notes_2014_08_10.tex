
\documentclass[prb,preprint]
{revtex4-1} 
% The line above defines the type of LaTeX document.
% Note that AJP uses the same style as Phys. Rev. B (prb).

% The % character begins a comment, which continues to the end of the line.

\usepackage{amsmath}  % needed for \tfrac, \bmatrix, etc.
\usepackage{amsfonts} % needed for bold Greek, Fraktur, and blackboard bold
\usepackage{graphicx} % needed for figures

\begin{document}

% Be sure to use the \title, \author, \affiliation, and \abstract macros
% to format your title page.  Don't use lower-level macros to  manually
% adjust the fonts and centering.

\title{EMII Notes 2014/08/10}
% In a long title you can use \\ to force a line break at a certain location.

\author{Hamilton B. Carter}
%\email{hcarter333@tamu.edu} % optional
% optional second address
% If there were a second author at the same address, we would put another 
% \author{} statement here.  Don't combine multiple authors in a single
% \author statement.
\affiliation{Department of Physics, Texas A\&M University, College Station, TX 77843}
% Please provide a full mailing address here.


% See the REVTeX documentation for more examples of author and affiliation lists.

\date{\today}

%\begin{abstract}


%\end{abstract}
% AJP requires an abstract for all regular article submissions.
% Abstracts are optional for submissions to the "Notes and Discussions" section.




%\maketitle % title page is now complete

%\newpage
%\section{Board 1}

%\begin{figure}[h!]
%\centering
%\includegraphics[width=5in]{board1_2014_06_24.jpg}
% Notice the width specification.  Photographs should normally have a
% resolution of approximately 300 pixels per inch when printed, that is,
% a total width of about 1000 pixels for a photo to be printed one column
% wide.  Note also that this included photo is in .jpg format even though 
% a .tiff version should be submitted for final production.
%\caption{Board 1)}
%\label{Board 1}
%\end{figure}
\centerline{\bf EMII Notes 2014/08/10}
\bigskip

Summary of what's gone on before.  The use of index notation to indicate a transpose was explained and shown with a concrete example.  Today, work on old homeworks begin.  The big issue of the day was figuring out an elegant way of showing that a contracted product was antisymmetric. The insights about commuting terms in index based products and how the Levi-Civita symbol works were worth the effort, but it was in fact a lot of effort!


The first part of the rotted cross product problem is to show that

$W_{il}W_{jm}W_{kn}\epsilon_{lmn} = det\left(W\right)\epsilon_{lmn}$

We're to do this by first showing that the left hand side is antisymmetric with respect to the $i$, $j$, and $k$ indices and therefore proportional to $\epsilon_{lmn}$ and then by showing that for a concreted example, $i=1$, $j=2$, $k=3$, the left hand side is equal to $det\left(W\right)$

There's a long way and a short was to show the antyisymmetry of the left hand side.\\


\textbf{The Short Way}

Swapping the first two indices of any two terms in the product and then commuting the terms is indistinguishable from swapping the corresponding summation, (second), indices in the two terms.  Heres an example:\\

Swap the first two indices and commute

$W_{il}W_{jm}W_{kn} \rightarrow W_{jl}W_{im}W_{kn} = W_{im}W_{jl}W_{kn}$
\\

Swap the summation indices

$W_{il}W_{jm}W_{kn} \rightarrow W_{im}W_{jl}W_{kn}$
\\

Swapping the summation indices however, swaps the two indices in the Levi-Civita symbol which changes the sign of the symbol.  This means that, interchanging the first indices of any two terms in the product introduces a negative sign and the product as a whole is anti-symmetric under the swap of $i$, $j$, or $k$.
\\

The Long Way

The second indices in each term are limited by the non-zero values of the Levi-Civita symbol that is contacted with the first three terms of the product.  We can fairly easily write out the six available combinations,\\
$W_{il}W_{jm}W_{kn}\epsilon_{lmn}$\\
$= W_{i1}W_{j2}W_{k3} + W_{i3}W_{j1}W_{k2} + W_{i2}W_{j3}W_{k1} - W_{i3}W_{j2}W_{k1} - W_{i2}W_{j1}W_{k3} - W_{i1}W_{j3}W_{k2}$

Number the above terms 1 through 6.  It is easy to show that by swapping $i$ and $j$ we get the following swaps, $1\leftrightarrow 6$,  $2\leftrightarrow 5$, and $3\leftrightarrow 4$.  \\

When the first term becomes the sixth term, there is a corresponding sign change.  The same thing happens for terms 2 and 5 as well as 3 and 4.
\\

{\bf Showing the Concrete determinant}\\

The second part of the problem is to show that for the index choices, $i = 1$, $j = 2$, $k = 3$, the l.h.s. is equal to $det\left(W\right)$.  This is just plug and chug work\\
Start with\\
$W_{il}W_{jm}W_{kn} \rightarrow W_{1l}W_{2m}W_{3n}$\\
There are six terms in the contracted sum\\
$= W_{11}W_{22}W_{33} + W_{13}W_{21}W_{32} + W_{12}W_{23}W_{31} - W_{13}W_{22}W_{31} - W_{12}W_{21}W_{33} - W_{11}W_{23}W_{32}$
\\
Now, work out the determinant of $W_{ij}$
\\
$det\left(W\right) = \begin{vmatrix}
W_{11} & W_{12} & W_{13}\\
W_{21} & W_{22} & W_{23}\\
W_{31} & W_{32} & W_{33}\\
\end{vmatrix}$\\\\
$=W_{11}W_{22}W_{33} - W_{11}W_{23}W_{32} - W_{12}W_{21}W_{33} + W_{12}W_{23}W_{31} + W_{13}W_{21}W_{32} - W_{13}W_{22}W_{31}$\\\\
The results are the same.


\bigskip
\textbf{The Many Missteps to the Short Way}

These are the attempts at the short way that didn't quite pan out just so I can see where I was misled.

To show the l.h.s. is antisymmetric, we need to show that by swapping any two of the $i$, $j$, and $k$ indices, we wind up with a quantity fo the same magnitude, but different components.  Let's start with the expression

$W_{il}W_{jm}W_{kn}\epsilon_{lmn}$,

and swap two of the indices, say $i$, and $j$.  This gives

$W_{jl}W_{im}W_{kn}\epsilon_{lmn}$.

Now, if we switch the l and m indices, the swap in the indices of the Levi-Civita symbol will produce a negative sign. Then, keeping in mind that in index notation, multiplication always commutes, we can write the above down as:

$W_{jl}W_{im}W_{kn}\epsilon_{lmn} = -W_{jm}W_{il}W_{kn}\epsilon_{lmn} = -W_{il}W_{jm}W_{kn}\epsilon_{mln}$.

New try

In all cases, the second index is restricted by the non-zero values of the Levi-Civita tensor.  If I switch the first indices on any two terms in the expression, then I've just effectively switched the second indices.  Switching the second indices can 


The swap of any two fo the first two indices is the same thing as swapping the second two indices for those two terms.  However, swapping the second two indices will introduce a negative sign from the Levi-Civita symbol.  This proves that swapping any two first indices is the same operation as introducing a negative sign.  An exmple may serve to make this more clear

$W_{il}W_{jm}W_{kn} \leftrightarrow W_{jl}W_{im}W_{kn}$


%\newpage
%\section{board 2}

%It looks like board 2 is just a better quality photo of board 1.

%\begin{figure}[h!]
%\centering
%\includegraphics[width=5in]{board2_2014_06_24.jpg}
% Notice the width specification.  Photographs should normally have a
% resolution of approximately 300 pixels per inch when printed, that is,
% a total width of about 1000 pixels for a photo to be printed one column
% wide.  Note also that this included photo is in .jpg format even though 
% a .tiff version should be submitted for final production.
%\caption{Board 2)}
%\label{Board 2}
%\end{figure}

%\newpage
%\section{board 3}

%\begin{figure}[h!]
%\centering
%\includegraphics[width=5in]{board3_2014_06_24.jpg}
% Notice the width specification.  Photographs should normally have a
% resolution of approximately 300 pixels per inch when printed, that is,
% a total width of about 1000 pixels for a photo to be printed one column
% wide.  Note also that this included photo is in .jpg format even though 
% a .tiff version should be submitted for final production.
%\caption{Board 3)}
%\label{Board 3}
%\end{figure}


%Board 13

%\begin{figure}[h!]
%\centering
%\includegraphics[width=5in]{board13_06_19_2014.JPG}
% Notice the width specification.  Photographs should normally have a
% resolution of approximately 300 pixels per inch when printed, that is,
% a total width of about 1000 pixels for a photo to be printed one column
% wide.  Note also that this included photo is in .jpg format even though 
% a .tiff version should be submitted for final production.
%\caption{Board 13}
%\label{Board 13}
%\end{figure}

%Board 14

%\begin{figure}[h!]
%\centering
%\includegraphics[width=5in]{board14_06_19_2014.JPG}
% Notice the width specification.  Photographs should normally have a
% resolution of approximately 300 pixels per inch when printed, that is,
% a total width of about 1000 pixels for a photo to be printed one column
% wide.  Note also that this included photo is in .jpg format even though 
% a .tiff version should be submitted for final production.
%\caption{Board 14}
%\label{Board 14}
%\end{figure}











%Board 4a

%\begin{figure}[h!]
%\centering
%\includegraphics[width=5in]{board4a_2014_06_12.jpg}
% Notice the width specification.  Photographs should normally have a
% resolution of approximately 300 pixels per inch when printed, that is,
% a total width of about 1000 pixels for a photo to be printed one column
% wide.  Note also that this included photo is in .jpg format even though 
% a .tiff version should be submitted for final production.
%\caption{Board 4a}
%\label{Board 4a}
%\end{figure}





% If your manuscript is conditionally accepted, the editors will ask you to
% submit your editable LaTeX source file.  Before doing so, you should move
% all tables and figure captions to the end, as shown below.  Tables come 
% first, followed by figure captions (with figure inclusions commented-out).
% Figures should be submitted as separate files, collected with the
% LaTeX file into a single .zip archive.

%\newpage   % Start a new page for tables

%\begin{table}[h!]
%\centering
%\caption{Elementary bosons}
%\begin{ruledtabular}
%\begin{tabular}{l c c c c p{5cm}}
%Name & Symbol & Mass (GeV/$c^2$) & Spin & Discovered & Interacts with \\
%\hline
%Photon & $\gamma$ & \ \ 0 & 1 & 1905 & Electrically charged particles \\
%Gluons & $g$ & \ \ 0 & 1 & 1978 & Strongly interacting particles (quarks and gluons) \\
%Weak charged bosons & $W^\pm$ & \ 82 & 1 & 1983 & Quarks, leptons, $W^\pm$, $Z^0$, $\gamma$ \\
%Weak neutral boson & $Z^0$ & \ 91 & 1 & 1983 & Quarks, leptons, $W^\pm$, $Z^0$ \\
%Higgs boson & $H$ & 126 & 0 & 2012 & Massive particles (according to theory) \\
%\end{tabular}
%\end{ruledtabular}
%\label{bosons}
%\end{table}

%\newpage   % Start a new page for figure captions

%\section*{Figure captions}

%\begin{figure}[h!]
%\centering
%\includegraphics{GasBulbData.eps}   % This line stays commented-out
%\caption{Pressure as a function of temperature for a fixed volume of air.  
%The three data sets are for three different amounts of air in the container. 
%For an ideal gas, the pressure would go to zero at $-273^\circ$C.  (Notice
%that this is a vector graphic, so it can be viewed at any scale without
%seeing pixels.)}

%\label{gasbulbdata}
%\end{figure}

%\begin{figure}[h!]
%\centering
%\includegraphics[width=5in]{ThreeSunsets.jpg}   % This line stays commented-out
%\caption{Three overlaid sequences of photos of the setting sun, taken
%near the December solstice (left), September equinox (center), and
%June solstice (right), all from the same location at 41$^\circ$ north
%latitude. The time interval between images in each sequence is approximately
%four minutes.}
%\label{sunsets}
%\end{figure}

\end{document}
