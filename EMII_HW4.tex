
\documentclass[prb,preprint]
{revtex4-1} 
% The line above defines the type of LaTeX document.
% Note that AJP uses the same style as Phys. Rev. B (prb).

% The % character begins a comment, which continues to the end of the line.

\usepackage{amsmath}  % needed for \tfrac, \bmatrix, etc.
\usepackage{amsfonts} % needed for bold Greek, Fraktur, and blackboard bold
\usepackage{graphicx} % needed for figures
\newcommand{\PRLsep}{\noindent\makebox[\linewidth]{\resizebox{0.8888\linewidth}{2pt}{$\bullet$}}\bigskip}

\begin{document}

% Be sure to use the \title, \author, \affiliation, and \abstract macros
% to format your title page.  Don't use lower-level macros to  manually
% adjust the fonts and centering.

\title{EMII Homework IV Due 2014/10/08}
% In a long title you can use \\ to force a line break at a certain location.

\author{Hamilton B. Carter}
\email{hcarter333@tamu.edu} % optional
% optional second address
% If there were a second author at the same address, we would put another 
% \author{} statement here.  Don't combine multiple authors in a single
% \author statement.
\affiliation{Department of Physics, Texas A\&M University, College Station, TX 77843}
% Please provide a full mailing address here.


% See the REVTeX documentation for more examples of author and affiliation lists.

\date{\today}

%\begin{abstract}


%\end{abstract}
% AJP requires an abstract for all regular article submissions.
% Abstracts are optional for submissions to the "Notes and Discussions" section.




\maketitle % title page is now complet

%\newpage
%\section{Board 1}

%\begin{figure}[h!]
%\centering
%\includegraphics[width=5in]{board1_2014_06_24.jpg}
% Notice the width specification.  Photographs should normally have a
% resolution of approximately 300 pixels per inch when printed, that is,
% a total width of about 1000 pixels for a photo to be printed one column
% wide.  Note also that this included photo is in .jpg format even though 
% a .tiff version should be submitted for final production.
%\caption{Board 1)}
%\label{Board 1}
%\end{figure}
%\centerline{\bf EMII Homework I Due 2014/09/17}
%\bigskip
\textbf{1}
\\
Write the Lorentz force equation $m \frac{d^2 x^\mu}{d\tau^2} = e F^\mu_{\;\nu} \frac{d x^\nu}{d\tau}$ as a matrix equation, where $F$ and $X$ are 
\\
$F = \begin{pmatrix}
F^0_{\;0} & F^0_{\;1} & F^0_{\;2} & F^0_{\;3}\\
F^1_{\;0} & F^1_{\;1} & F^1_{\;2} & F^1_{\;3}\\
F^2_{\;0} & F^2_{\;1} & F^2_{\;2} & F^2_{\;3}\\
F^3_{\;0} & F^3_{\;1} & F^3_{\;2} & F^3_{\;3}\\
\end{pmatrix} $, 
$X = \begin{pmatrix}
x^0\\
x^1\\
x^2\\
x^3\\
\end{pmatrix} $
\\
\\
Obtain the general solution for $X\left(\tau\right)$ as a simple matrix expression.
\\
\\
First, we can write the Lorentz force expression in matrix form,
\\
\\
$m \dfrac{d^2 X}{d\tau^2} = e F \dfrac{dX}{d\tau}$.
\\
\\
Then, we integrate with respect to $\tau$ once to get,
\\
\\
$\dfrac{d X}{d\tau} = \dfrac{e F}{m} X\left(\tau\right) + C$.
\\
\\
Where $C$ are initial conditions with units of proper velocity
\\
The solution we'll try is an ansatz of the form 
\\
\\
$X\left(\tau\right) = e^{\frac{e}{m}F\tau}X\left(0\right)$
\\
\\
Where, $X\left(0\right)$ is the intial state of the position matrix $X$.  
\\
\\
To check the solution, we'll make use of the definition of exponentiation for a matrix given in the assignment:
\\
\\
$e^{\frac{e}{m}F\tau} = \dfrac{e}{m}F^0 + \dfrac{\tau}{1}\dfrac{e}{m}A^1 + \dfrac{\tau}{2!}\left(\dfrac{e}{m}A\right)^2 + ...$
\\
\\
When we take the derivative of the expression, since $\dfrac{e}{m}F$ is not a function of $\tau$, we get
\\
\\
$\dfrac{d}{d\tau} e^{A\tau} = \left(0 + \dfrac{t}{1!}\dfrac{e}{m}F + \dfrac{2t^2}{2!}\left(\dfrac{e}{m}F\right)^2 + \dfrac{3t^3}{3!}\left(\dfrac{e}{m}F\right)^2\right)\dfrac{e}{m}F$
\\
\\
$\dfrac{d}{d\tau} e^{\frac{e}{m}F} = \dfrac{e}{m}F\;e^{\frac{e}{m}F}$ 
\\
\\
\PRLsep
\\
\newpage
\textbf{2.a.}
\\
We're to take the definition of the Hodge dual
$*F_{\mu\nu} = \dfrac{1}{2}\epsilon_{\mu\nu\rho\sigma}F^{\mu\nu}$ and show that
\\
\\
$\partial^\mu *F_{\mu\nu} = 0$
\\
\\
in analogy to our original Bianchi identity
\\
\\
$\partial_\mu F_{\nu\rho} + \partial_\nu F_{\rho\mu} + \partial_\rho F_{\mu\nu}$
\\
\\
First, notice that there are degnerate cases in the analogous expression that will make it automatically go to zero.  These are the cases where $\mu = \nu$, $\nu = \rho$, and $\rho = \mu$ because $F_{\mu\nu}$ is assymetric.
\\
\\
These same cases also appear in the first expression, where if any of the indeics of $\epsilon_{\mu\nu\rho\sigma}$ are equal to each other then the epxression is zero.
\\
\\
$\nu$ is the only free index of the Hodge dual expression above.  Once that index is chosen, then $\mu$ is summed over only three values.  Each of these three values locks in only two possible values for the $F^{\mu\nu}$ expression.  Since the last two indeces on the two index field four-tensor are anti-symmetric, and the four dimensional Levi-Civita is anti-symmetric, swapping these last two indeices results in two times the magnitude of either term.  So, looking at the case where $\nu = 0$ for example, we wind up with 
\\
\\
$\left(\partial^1 F_{23} + \partial^2 F_{13} +\partial^3 F_{12}\right)$
\\
\\
We can see that this produces three terms as in our original Bianchi identity that follow the same index rules that our three indices, (in this case, labeled as $\mu$, $\rho$, and $\sigma$), $\mu \neq \rho \neq \sigma$.  By cycling over all the available indices he Hodge dual derivative expression produces the exact same terms as the analogous Bianchi identity we started with.
\\
\PRLsep
\newpage
\textbf{2.b.}
\\
In this problem we'll use the Hodge dual identity proven above, namely, 
\\
\\
$\partial^\mu *F_{\mu\nu} = \partial^\mu \dfrac{1}{2}\epsilon_{\mu\nu\rho\sigma}F^{\mu\nu} = 0$
\\
\\
To show that 
\\
\\
$\partial_\mu V^\mu = \dfrac{1}{2}\epsilon^{\mu\nu\rho\sigma}F_{\mu\nu}F_{\rho\sigma}$
\\
\\
where $V^\mu = \epsilon^{\mu\nu\rho\sigma}A_\nu F_{\rho\sigma}$
\\
\\
The first step is to evluate $\partial_\mu V^\mu$ to get
\\
\\
$\partial_\mu V^\mu = \epsilon^{\mu\nu\rho\sigma}\partial_\mu A_\nu F_{\rho\sigma} + A_\nu \partial_\mu\epsilon^{\mu\nu\rho\sigma} F_{\rho\sigma}$
\\
\\
Via the Hodge dual identity stated above, the second term evaluates to zero leaving us with 
\\
\\
$\partial_\mu V^\mu = \epsilon^{\mu\nu\rho\sigma}\partial_\mu A_\nu F_{\rho\sigma}$
\\
\\
Or... We could do it the easy way.  We could use the expression 
\\
\\
$F_{\mu\nu} = \partial_\mu A_\nu - \partial_\nu A_\mu$ 
\\
\\
to rewrite the right hand side of what we want to prove:
\\
\\
$\partial_\mu \epsilon^{\mu\nu\rho\sigma}A_\nu F_{\rho\sigma} = \dfrac{1}{2}\epsilon^{\mu\nu\rho\sigma}F_{\mu\nu}F_{\rho\sigma}$
\\
to get, 
\\
\\
$\dfrac{1}{2}\epsilon^{\mu\nu\rho\sigma} \left(\partial_\mu A_\nu - \partial_\nu A_\mu\right)F_{\rho\sigma}$
\\
\\
Distributing, we get
\\
\\
$\dfrac{1}{2} \left(\epsilon^{\mu\nu\rho\sigma} \partial_\mu A_\nu F_{\rho\sigma} - \epsilon^{\mu\nu\rho\sigma} \partial_\nu A_\mu F_{\rho\sigma}\right)$
\\
\\
Rearranging this just a tad gives:
\\
\\
$\dfrac{1}{2} \left(\partial_\mu \epsilon^{\mu\nu\rho\sigma}  A_\nu F_{\rho\sigma} - \epsilon^{\mu\nu\rho\sigma} \partial_\nu A_\mu F_{\rho\sigma}\right)$
\\
\\
Where we realize that the first term in parentheis is identical to the left hand of the expression, now, we just need to convert the minus sign on the second term to a plus sign, and we're done.  Interchanging the indices on the partial and the field four-vector on the second term introduces a negative sign:
\\
\\
$\dfrac{1}{2} \left(\partial_\mu \epsilon^{\mu\nu\rho\sigma}  A_\nu F_{\rho\sigma} + \epsilon^{\mu\nu\rho\sigma} \partial_\mu A_\nu F_{\rho\sigma}\right) = \partial_\mu \epsilon^{\mu\nu\rho\sigma}A_\nu F_{\rho\sigma}$
\\
\\
Which is equal to the left hand side of the expression we were trying to prove!
\\
\PRLsep
\newpage
\textbf{3.a.}
\\
Show that $\epsilon^{\mu\nu\rho\sigma}\epsilon_{\alpha\beta\gamma\sigma} = \delta^\mu_\alpha \delta^\nu_\beta \delta^\rho_\gamma - \delta^\nu_\alpha \delta^\rho_\beta \delta^\mu_\gamma - \delta^\rho_\alpha \delta^\mu_\beta \delta^\nu_\gamma + \delta^\nu_\alpha \delta^\mu_\beta \delta^\rho_\gamma + \delta^\mu_\alpha \delta^\rho_\beta \delta^\nu_\gamma + \delta^\rho_\alpha \delta^\nu_\beta \delta^\mu_\gamma$ 
\\
\\
Proceed exactly as in the proof for the three index version.  Write out the sum:
\\
\\
$\epsilon^{\mu\nu\rho 0}\epsilon_{\alpha\beta\gamma 0} + \epsilon^{\mu\nu\rho 1}\epsilon_{\alpha\beta\gamma 1} + \epsilon^{\mu\nu\rho 2}\epsilon_{\alpha\beta\gamma 2} + \epsilon^{\mu\nu\rho 3}\epsilon_{\alpha\beta\gamma 3}$
\\
\\
From the properties of the tensor, you can see that if we choose, $\mu = 1$, $\nu = 2$, and $\rho = 3$, or any combination thereof to make the first term not equal zero, the other three terms will automatically evalue to zero. The same can be said of any of the terms, so at most, we need to deal with one of the four terms, and we'll continue to work with the first one.
\\
\\
For this expression, the only index choices that will evaluate as non-zero are shown in the table below:
\begin{table}[h!]
\centering
\caption{Index choices and Results}
\begin{ruledtabular}
\begin{tabular}{l c c c c c c p{5cm}}
% The codes above determine the horizontal alignment in each column.
% Options are l (left), r (right), c (centered), and p (paragraph).
% The p option allows an entry to be broken into multiple lines, and
% therefore requires a width specification, in this case 5 centimeters.
$\alpha$ & $\beta$ & $\gamma$ & $\mu$ & $\nu$ & $\rho$ & result \\
\hline	% horizontal line to separate headings from data
1 & 2 & 3 & 1 & 2 & 3 & +1 \\
1 & 2 & 3 & 1 & 3 & 2 & -1 \\
1 & 2 & 3 & 2 & 1 & 3 & -1 \\
1 & 2 & 3 & 2 & 3 & 1 & +1 \\
1 & 2 & 3 & 3 & 1 & 2 & +1 \\
1 & 2 & 3 & 3 & 2 & 1 & -1 \\
\end{tabular}
\end{ruledtabular}
\label{bosons}
\end{table}
\\
Looking at the right hand side of the expredssion we want to prove, only of the six terms can evaluate to non-zero.  Once the index choice has been made for that term, the other five terms go to zero, (I'm repeating myself).  The table can be expressed by permuting the indices $\mu$, $\nu$, and $\rho$ instead of their values to arrive at
\\
\\
$\epsilon^{\mu\nu\rho\sigma}\epsilon_{\alpha\beta\gamma\sigma} = \delta^\mu_\alpha \delta^\nu_\beta \delta^\rho_\gamma - \delta^\nu_\alpha \delta^\rho_\beta \delta^\mu_\gamma - \delta^\rho_\alpha \delta^\mu_\beta \delta^\nu_\gamma + \delta^\nu_\alpha \delta^\mu_\beta \delta^\rho_\gamma + \delta^\mu_\alpha \delta^\rho_\beta \delta^\nu_\gamma + \delta^\rho_\alpha \delta^\nu_\beta \delta^\mu_\gamma$
\\
\PRLsep
\\
\\
\newpage
\textbf{3.c.}
\\
Notes on how to approach the problem.  Notice that the experssion in 3.c. comes from Dirac's article.  Also keep in mind that $M_{\mu\nu}$ looks like a cross product and also looks like a similarity trnaformation once applied.  Also, it looks like the answer for the last part of 3a is contained in 3c. 
\\
\\
Given that $M_{\mu\nu}$ is defined as 
\\
$M_{\mu\nu} = x_\mu\partial_\nu - x_\nu\partial_\mu$
\\
Show that 
\\
$\dfrac{1}{2} M_{\rho\sigma}\left(x^\mu\right) = -\lambda^\mu_{\;\nu} x^\nu = -\delta x^\mu$
\\
First, write out $\frac{1}{2}M_{\rho\sigma}\left(x^\mu\right)$
\\
$\dfrac{1}{2}\lambda^{\rho\sigma}\left(x_\rho \partial_\sigma x^\mu - x_\sigma \partial_\rho x^\mu\right)$
\\
then, evaluate the partial derivatives
\\
$\dfrac{1}{2}\lambda^{\rho\sigma}\left(x_\rho \delta^\mu_\sigma - x_\sigma \delta^\mu_\rho\right)$
\\
Multiply the common factor back in
\\
$= \dfrac{1}{2}\lambda^{\sigma\mu}x_\sigma - \dfrac{1}{2}\lambda^{\mu\sigma}x_\sigma$
\\
Which, from the anti-symmetry of $\lambda_{\mu\nu}$ shown above, produces
\\
$=\lambda^{\sigma\mu} x_\sigma$
\\
$=-\lambda^{\mu\sigma} x_\sigma$
\\
We can always swap the vertical positions of the dummy indices.
\\
$=-\lambda^\mu_{\;\sigma} x^\sigma = -\delta x^\sigma$
\\
The last equality is demonstrated in 3.a. above.
\\
\\
\PRLsep
\\
\\
\newpage
\textbf{3.d.}
\\
Show that $M_{\mu\nu}$ obeys the algebra, 
$\left[M_{\mu\nu}, M_{\rho\sigma}\right] = eta_{\nu\rho}M_{\mu\sigma} - eta_{\mu\rho}M_{\nu\sigma} + eta_{\mu\sigma}M_{\nu\rho} - eta_{\nu\sigma}M_{\mu\rho}$
\\
Do the obvious thing and evaluate the commutator which will give 8 terms.
\\
$\left(x_\mu \partial_\nu - x_\nu \partial_\mu\right)\left(x_\rho \partial_\sigma - x_\sigma \partial_\rho\right)$
\\
The resulting 8 terms are listed and simplified below
\\
1.  $x_\mu \partial_\nu x_\rho \partial_\sigma = x_\mu\partial_\nu\eta{\rho_\alpha}x^\alpha \partial_\sigma$
\\
The above uses the identity $U_\mu = \eta_{\mu\nu}U^\nu$.  Note that the dummy index goes to the rigth on the $\eta$.
\\
Getting started with number 1 again, we have,
\\
1.  $x_\mu \partial_\nu x_\rho \partial_\sigma = x_\mu\partial_\nu\eta{\rho_\alpha}x^\alpha \partial_\sigma = x_\mu \delta_\nu^\alpha \eta_{\rho\alpha} = x_\mu \eta_{\rho\nu} \partial_\sigma = \eta_{\rho\nu} x_\mu \partial_\sigma$
\\
The remaining seven terms will be written out following the example of the first term
\\
2.  $-x_\nu \partial_\mu x_\rho \partial_\sigma = \eta_{\rho\mu} x_\nu \partial_\sigma$
\\
3.  $x_\nu \partial_\mu x_\sigma \partial_\rho = \eta_{\sigma\mu} x_\nu \partial_\rho$
\\
4.  $-x_\mu \partial_\nu x_\sigma \partial_\rho = \eta_{\sigma\nu} x_\mu \partial_\rho$
\\
5.  $x_\rho \partial_\sigma x_\mu \partial_\nu = \eta_{\mu\sigma} x_\rho \partial_\nu$
\\
6.  $x_\rho \partial_\sigma x_\nu \partial_\mu = \eta_{\nu\sigma} x_\rho \partial_\mu$
\\
7.  $-x_\sigma \partial_\rho x_\nu \partial_\mu = \eta_{\nu\rho} x_\sigma \partial_\mu$
\\
8.  $x_\sigma \partial_\rho x_\mu \partial_\nu = \eta_{\mu\rho} x_\sigma \partial_\nu$
\\
Rearraning terms, grouping, and adding, we see that we can get the desired result by the sum of grouped terms, 
\\
(1 and 7) + (2 and 8) + (3 and 5) + (4 and 6)
\\
$\eta_{\nu\rho}M_{\mu\sigma} - \eta_{\mu\rho}M_{\nu\sigma} + \eta_{\mu\sigma}M_{\nu\rho} - \eta_{\nu\sigma}M_{\nu\rho}$



% If your manuscript is conditionally accepted, the editors will ask you to
% submit your editable LaTeX source file.  Before doing so, you should move
% all tables and figure captions to the end, as shown below.  Tables come 
% first, followed by figure captions (with figure inclusions commented-out).
% Figures should be submitted as separate files, collected with the
% LaTeX file into a single .zip archive.

%\newpage   % Start a new page for tables

%\begin{table}[h!]
%\centering
%\caption{Elementary bosons}
%\begin{ruledtabular}
%\begin{tabular}{l c c c c p{5cm}}
%Name & Symbol & Mass (GeV/$c^2$) & Spin & Discovered & Interacts with \\
%\hline
%Photon & $\gamma$ & \ \ 0 & 1 & 1905 & Electrically charged particles \\
%Gluons & $g$ & \ \ 0 & 1 & 1978 & Strongly interacting particles (quarks and gluons) \\
%Weak charged bosons & $W^\pm$ & \ 82 & 1 & 1983 & Quarks, leptons, $W^\pm$, $Z^0$, $\gamma$ \\
%Weak neutral boson & $Z^0$ & \ 91 & 1 & 1983 & Quarks, leptons, $W^\pm$, $Z^0$ \\
%Higgs boson & $H$ & 126 & 0 & 2012 & Massive particles (according to theory) \\
%\end{tabular}
%\end{ruledtabular}
%\label{bosons}
%\end{table}

%\newpage   % Start a new page for figure captions

%\section*{Figure captions}

%\begin{figure}[h!]
%\centering
%\includegraphics{GasBulbData.eps}   % This line stays commented-out
%\caption{Pressure as a function of temperature for a fixed volume of air.  
%The three data sets are for three different amounts of air in the container. 
%For an ideal gas, the pressure would go to zero at $-273^\circ$C.  (Notice
%that this is a vector graphic, so it can be viewed at any scale without
%seeing pixels.)}

%\label{gasbulbdata}
%\end{figure}

%\begin{figure}[h!]
%\centering
%\includegraphics[width=5in]{ThreeSunsets.jpg}   % This line stays commented-out
%\caption{Three overlaid sequences of photos of the setting sun, taken
%near the December solstice (left), September equinox (center), and
%June solstice (right), all from the same location at 41$^\circ$ north
%latitude. The time interval between images in each sequence is approximately
%four minutes.}
%\label{sunsets}
%\end{figure}

\end{document}
