
\documentclass[prb,preprint]
{revtex4-1} 
% The line above defines the type of LaTeX document.
% Note that AJP uses the same style as Phys. Rev. B (prb).

% The % character begins a comment, which continues to the end of the line.

\usepackage{amsmath}  % needed for \tfrac, \bmatrix, etc.
\usepackage{amsfonts} % needed for bold Greek, Fraktur, and blackboard bold
\usepackage{graphicx} % needed for figures

\begin{document}

% Be sure to use the \title, \author, \affiliation, and \abstract macros
% to format your title page.  Don't use lower-level macros to  manually
% adjust the fonts and centering.

\title{EMII Notes 2014/08/20}
% In a long title you can use \\ to force a line break at a certain location.

\author{Hamilton B. Carter}
%\email{hcarter333@tamu.edu} % optional
% optional second address
% If there were a second author at the same address, we would put another 
% \author{} statement here.  Don't combine multiple authors in a single
% \author statement.
\affiliation{Department of Physics, Texas A\&M University, College Station, TX 77843}
% Please provide a full mailing address here.


% See the REVTeX documentation for more examples of author and affiliation lists.

\date{\today}

%\begin{abstract}


%\end{abstract}
% AJP requires an abstract for all regular article submissions.
% Abstracts are optional for submissions to the "Notes and Discussions" section.




%\maketitle % title page is now complete

%\newpage
%\section{Board 1}

%\begin{figure}[h!]
%\centering
%\includegraphics[width=5in]{board1_2014_06_24.jpg}
% Notice the width specification.  Photographs should normally have a
% resolution of approximately 300 pixels per inch when printed, that is,
% a total width of about 1000 pixels for a photo to be printed one column
% wide.  Note also that this included photo is in .jpg format even though 
% a .tiff version should be submitted for final production.
%\caption{Board 1)}
%\label{Board 1}
%\end{figure}
\centerline{\bf EMII Notes 2014/08/20}
\bigskip

Summary:  Still More Tensor Identities\\
Later today, the special relativity section begins!  That'll be fun!
\\
\\
$\vec{A} \cdot \left(\vec{B} \times \vec{C}\right) = \vec{B} \cdot \left(\vec{C} \times \vec{A}\right)$
\\
$= A_i \epsilon_{ijk} B_j C_k$
\\
As long as we only cycle the indices in the Levi-Civita symbol, we won't cause a sign change, so the above is also equal to 
\\
$= A_k \epsilon_{ijk} B_i C_j$
\\
Which we can commute to get 
\\
$= B_i \epsilon_{ijk} C_j A_k = \vec{B} \cdot \left(\vec{C} \times \vec{A}\right)$
\\\\
Done!
\\\\



$\vec{\nabla} \cdot \left(\vec{\nabla} \times \vec{A} \right) = 0$
\\
$=\partial_i \epsilon_{ijk} \partial_j A_k$\\
$= 0$

If $i$ and $j$ are equal, then the Levi-Civita evaluates to zero.  If they are not equal, then swapping the two indices produces the same mixed partial derivative result, but with a negative sign inserted by swapping indices in the Levi-Civita symbol.  These equal but opposite terms all sum to zero giving the advertised result.
\\\\

$\vec{\nabla} \times \vec{\nabla} \times \vec{A} = \vec{\nabla}\left(\vec{\nabla} \cdot \vec{A}\right) - \nabla^2 \vec{A}$
\\
$= \epsilon_{lmi} \partial_m \epsilon_{ijk} \partial_j A_k$
\\
Don't let all the dels, nablas, whatever you'd like to call them, throw you off.  At the end of the day, this is just the same as the $\vec{A} \times \vec{B} \times \vec{C}$ example.  Use the two Levi-Civitas to four deltas identity.\\
$= \epsilon_{lmi} \partial_m \epsilon_{ijk} \partial_j A_k = \epsilon_{ijk} \partial_j \epsilon_{lmk} \partial_l A_m  = \left(\delta_{il}\delta_{jm} - \delta_{im}\delta_{jl}\right)\partial_j\partial_l A_m$
\\
$=\delta_{il}\delta_{jm}\partial_j\partial_l A_m - \delta_{im}\delta_{jl}\partial_j\partial_l A_m$
\\
$= \partial_i \partial_j A_j - \partial_j \partial_j A_i = \vec{\nabla}\left(\vec{\nabla}\cdot \vec{A}\right) - \nabla^2 \vec{A}$






%\newpage
%\section{board 2}

%It looks like board 2 is just a better quality photo of board 1.

%\begin{figure}[h!]
%\centering
%\includegraphics[width=5in]{board2_2014_06_24.jpg}
% Notice the width specification.  Photographs should normally have a
% resolution of approximately 300 pixels per inch when printed, that is,
% a total width of about 1000 pixels for a photo to be printed one column
% wide.  Note also that this included photo is in .jpg format even though 
% a .tiff version should be submitted for final production.
%\caption{Board 2)}
%\label{Board 2}
%\end{figure}

%\newpage
%\section{board 3}

%\begin{figure}[h!]
%\centering
%\includegraphics[width=5in]{board3_2014_06_24.jpg}
% Notice the width specification.  Photographs should normally have a
% resolution of approximately 300 pixels per inch when printed, that is,
% a total width of about 1000 pixels for a photo to be printed one column
% wide.  Note also that this included photo is in .jpg format even though 
% a .tiff version should be submitted for final production.
%\caption{Board 3)}
%\label{Board 3}
%\end{figure}


%Board 13

%\begin{figure}[h!]
%\centering
%\includegraphics[width=5in]{board13_06_19_2014.JPG}
% Notice the width specification.  Photographs should normally have a
% resolution of approximately 300 pixels per inch when printed, that is,
% a total width of about 1000 pixels for a photo to be printed one column
% wide.  Note also that this included photo is in .jpg format even though 
% a .tiff version should be submitted for final production.
%\caption{Board 13}
%\label{Board 13}
%\end{figure}

%Board 14

%\begin{figure}[h!]
%\centering
%\includegraphics[width=5in]{board14_06_19_2014.JPG}
% Notice the width specification.  Photographs should normally have a
% resolution of approximately 300 pixels per inch when printed, that is,
% a total width of about 1000 pixels for a photo to be printed one column
% wide.  Note also that this included photo is in .jpg format even though 
% a .tiff version should be submitted for final production.
%\caption{Board 14}
%\label{Board 14}
%\end{figure}











%Board 4a

%\begin{figure}[h!]
%\centering
%\includegraphics[width=5in]{board4a_2014_06_12.jpg}
% Notice the width specification.  Photographs should normally have a
% resolution of approximately 300 pixels per inch when printed, that is,
% a total width of about 1000 pixels for a photo to be printed one column
% wide.  Note also that this included photo is in .jpg format even though 
% a .tiff version should be submitted for final production.
%\caption{Board 4a}
%\label{Board 4a}
%\end{figure}





% If your manuscript is conditionally accepted, the editors will ask you to
% submit your editable LaTeX source file.  Before doing so, you should move
% all tables and figure captions to the end, as shown below.  Tables come 
% first, followed by figure captions (with figure inclusions commented-out).
% Figures should be submitted as separate files, collected with the
% LaTeX file into a single .zip archive.

%\newpage   % Start a new page for tables

%\begin{table}[h!]
%\centering
%\caption{Elementary bosons}
%\begin{ruledtabular}
%\begin{tabular}{l c c c c p{5cm}}
%Name & Symbol & Mass (GeV/$c^2$) & Spin & Discovered & Interacts with \\
%\hline
%Photon & $\gamma$ & \ \ 0 & 1 & 1905 & Electrically charged particles \\
%Gluons & $g$ & \ \ 0 & 1 & 1978 & Strongly interacting particles (quarks and gluons) \\
%Weak charged bosons & $W^\pm$ & \ 82 & 1 & 1983 & Quarks, leptons, $W^\pm$, $Z^0$, $\gamma$ \\
%Weak neutral boson & $Z^0$ & \ 91 & 1 & 1983 & Quarks, leptons, $W^\pm$, $Z^0$ \\
%Higgs boson & $H$ & 126 & 0 & 2012 & Massive particles (according to theory) \\
%\end{tabular}
%\end{ruledtabular}
%\label{bosons}
%\end{table}

%\newpage   % Start a new page for figure captions

%\section*{Figure captions}

%\begin{figure}[h!]
%\centering
%\includegraphics{GasBulbData.eps}   % This line stays commented-out
%\caption{Pressure as a function of temperature for a fixed volume of air.  
%The three data sets are for three different amounts of air in the container. 
%For an ideal gas, the pressure would go to zero at $-273^\circ$C.  (Notice
%that this is a vector graphic, so it can be viewed at any scale without
%seeing pixels.)}

%\label{gasbulbdata}
%\end{figure}

%\begin{figure}[h!]
%\centering
%\includegraphics[width=5in]{ThreeSunsets.jpg}   % This line stays commented-out
%\caption{Three overlaid sequences of photos of the setting sun, taken
%near the December solstice (left), September equinox (center), and
%June solstice (right), all from the same location at 41$^\circ$ north
%latitude. The time interval between images in each sequence is approximately
%four minutes.}
%\label{sunsets}
%\end{figure}

\end{document}
